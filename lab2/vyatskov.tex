\documentclass[11pt,a4paper,oneside]{article}

%\usepackage{pscyr}
\usepackage[T2A]{fontenc}
\usepackage[utf8x]{inputenc}
\usepackage[english,russian]{babel}
\usepackage{expdlist}
\usepackage[dvips]{graphicx}
\usepackage{amsmath}
\usepackage[makeroom]{cancel}
\usepackage{svg}
\usepackage[top=1in, bottom=1in, left=1in, right=1in]{geometry}
\usepackage{indentfirst}

\begin{document}

\begin{center}
	{Вяцков Михаил, КН-401}
	
	{\huge \bf Лабораторная работа №2}
\end{center}

\section{Условия}

$$ f(x) = \cos(3x) - x^3 = 0 $$
$$ f'(x) = -3 (\sin(3x) + x^2) $$
$$ \varepsilon = 0.5 \cdot 10^{-5} $$

\section{Корень}

По условиям задачи необходимо найти наименьший полжительный корень уравнения $f(x) = 0$. Определим аналитически, где этот корень находится. Докажем, что производная $f'(x)$ всегда отрицательна при $x > 0$. Действительно,

$$ x > 1 \implies x^2 > 1 \ge |\sin(3x)| \ge - \sin(3x) \implies x^2 + \sin(3x) > 0
	\implies f'(x) < 0 $$
$$ x = 1 \implies f'(x) = -3 (\sin(3) + 1) \approx -3.4 < 0 $$
$$ 0 < x < 1 \implies \sin(3x) > 0 \implies \sin(3x) + x^2 > 0 \implies f'(x) < 0 $$

Кроме того,

$$ f(0) = \cos(0) = 1 > 0 $$
$$ f\left(\frac{\pi}{6}\right) = \cos\left(\frac{\pi}{2}\right) - \left(\frac{pi}{6}\right)^3
	= - \left(\frac{pi}{6}\right)^3 < 0 $$
	
Таким образом, фукнция $f(x)$ монотонно убывает от 1 до $-\infty$, причем при $x = 0$ она принимает положительное значение, а при $x = \frac{\pi}{6}$ --- отрицательное, значит единственный положительный корень уравнения находится на отрезке $[0, \frac{\pi}{6}]$.

Заметим также, что $f(\frac{5 \pi}{36}) \approx 0.18 > 0$, поэтому мы можем даже взять $\alpha = [\frac{5 \pi}{36}, \frac{pi}{6}]$.

На данном отрезке производная монотонно убывает, поэтому мы можем найти как наибольшее, там и наименьшее по модулю значение производной.

$$ m = \min_{x \in \alpha}(|f'(x)|) = \left|f'\left(\frac{5 \pi}{36}\right)\right| > 3 $$
$$ M = \max_{x \in \alpha}(|f'(x)|) = \left|f'\left(\frac{\pi}{6}\right)\right| < 4 $$

\section{Метод подвижных хорд}

Метод подвижных хорд заключаеся в том, что мы берем две последние точки из построения $(x_n, f(x_n))$ и $(x_{n - 1}, f(x_{n - 1}))$ и строим по ним следующую, проводя хорду и пересекая ее с осью абсцисс. Таким образом следующая точка образуется по формуле

$$ x_{n + 1} = x_n - \frac{f(x_n) (x_n - x_{n - 1})}{f(x_n) - f(x_{n - 1})} $$

Выход из цикла в данном случае будет из условия

$$ \frac{f(x_n)}{\min_{x \in \alpha}(|f'(x)|)} = \frac{f(x_n)}{m} \le \varepsilon $$

\section{Метод простых итераций}

Напомним, что для МПИ составляется вспомогательная функция $\varphi(x)$ и решается уравнение $\varphi(x) = x$, при том условии, что производная $\varphi(x)$ на некоторой окрестности корня, откуда берется начальное приближение, меньше 1 по модулю. Тогда последовательность приближений строится таким образом

$$ x_{n + 1} = \varphi(x_n) $$

Заметим, что если мы положим сразу

$$ \varphi(x) = x + \cos(3x) - x^3 = x + f(x) $$

То получим для производной

$$ \varphi'(x) = 1 - 3 (\sin(3x) + x^2) = 1 + f'(x) $$

Но $m > 1$, а значит на всем рассматриваемо отрезке производная больше единицы, значит в таком виде МПИ неприменим. Зададим тогда параметр $0 < q < 1$ такой, что

$$ \widetilde{\varphi}(x) = 1 + q (\cos(3x) - x^3) $$
$$ \widetilde{\varphi}'(x) = 1 - 3 q (\sin(3x) + x^2) $$

Нам теперь необходимо задать параметр $q$ такой, чтобы производная в окрестности корня, например на $\alpha$, была меньше единицы по модулю. Как мы знаем, модуль производной $f$ заключен в отрезке от 3 до 4, поэтому взяв $q \le \frac{1}{4}$, получим, что модуль производной заключен в интервале от 0.75 до 1.

Объясним теперь, на что влияет этот параметр. Как мы знаем, МПИ использует тот факт, что после $n$ шагов выполняется следующее соотношение, с учетом того, что $\zeta$ --- корень, а длина отрезка $\alpha$ меньше 1

$$ |x_n - \zeta| \le \left( \max_{x \in \alpha}(\widetilde{\varphi}'(x)) \right)^{n} |x_0 - \zeta|
	= \left(1 - qm\right)^n |x_0 - \zeta| \le (1 - qm)^n $$
	
Теперь, положив

$$ (1 - qm)^n \le \varepsilon $$

Мы получаем, что достигли необходимой точности. Заметим, что $1 - qm < 1$ и следовательно $ln(1 - qm) < 0$, отсюда получаем соотношение

$$ n \ge \frac{\ln(\varepsilon)}{\ln(1 - qm)} $$

Рассматривая это как функцию от $q$ понимаем, что с приближением $q$ к нулю $1 - qm$ монотонно возрастает до 1 снизу, следовательно $ln(1 - qm)$ монотонно возрастает до нуля от $-\infty$, следовательно все выражение монотонно возрастает до бесконечности. Следовательно, чем дальше $q$ от нуля, тем меньше надо итераций, чтобы достичь необходимой точности. Так как если $q \ge \frac{1}{M}$, то метод разойдется, положим для удобства $q = \frac{1}{4}$

$$ n \ge \frac{\ln(0.5 \cdot 10^{-5})}{\ln(1 - \frac{3}{4} (sin(3 \frac{5 \pi}{36})+(\frac{5 \pi}{36})^2))} \approx 6 $$

То есть теоретически понадобится всего 6 итераций. Однако при практическом вычислении ограничим функцию также, как в прошлом методе, возможно потребуется даже меньше итераций.

\section{Численный эксперимент}

Ради эксперимента я попробовал взять точки пошире для метода подвижных хорд и другой коээфицент для МПИ

\begin{tabular}{ | l | l | l | l | l | }
	\hline
	Метод
		& $x_0 (x_1)$
		& Теоретическое $n$
		& Фактическое $n$
		& Полученный корень \\ \hline
	Подвижных хорд
		& $\frac{5 \pi}{36} \left(\frac{\pi}{6}\right)$
		& 
		& 3
		& \underline{0.48539}34311 \\ \hline
	Подвижных хорд
		& $0 \left(1\right)$
		& 
		& 47
		& \underline{0.48539}09869 \\ \hline
	Подвижных хорд
		& $0 \left(6\right)$
		& 
		& 12
		& Разошелся \\ \hline
	Подвижных хорд
		& $0.5 \left(6\right)$
		& 
		& 3
		& \underline{0.48539}50467 \\ \hline
	МПИ $(q = 0.25)$
		& $\frac{\pi}{6}$
		& 4
		& 4
		& \underline{0.48539}54080 \\ \hline
	МПИ $(q = 0.01)$
		& 6
		& ???
		& 347
		& \underline{0.48539}82727 \\ \hline
	МПИ $(q = 0.01)$
		& $\frac{\pi}{6}$
		& 314
		& 243
		& \underline{0.48539}83771 \\ \hline
\end{tabular}

\section{Выводы}

Методы сошлись чрезвычайно быстро при выборе достаточно близких точек из аналитических соображений. Если брать более широкие интервалы, то метод подвижных хорд может очень быстро разойтись при неправильном выборе точек, а выбор достаточно большого $q$ хоть и увеличивает количество шагов МПИ, но позволяет получить хоть ответ на широком наборе входных данных.

\end{document}
