\documentclass[11pt,a4paper,oneside]{article}

%\usepackage{pscyr}
\usepackage[T2A]{fontenc}
\usepackage[utf8x]{inputenc}
\usepackage[english,russian]{babel}
\usepackage{expdlist}
\usepackage[dvips]{graphicx}
\usepackage{amsmath}
\usepackage[makeroom]{cancel}
\usepackage{svg}
\usepackage[top=1in, bottom=1in, left=1in, right=1in]{geometry}

\begin{document}

\begin{center}
	{Вяцков Михаил, КН-401}
	
	{\huge \bf Лабораторная работа №1}
\end{center}

По условиям лабораторной

$$ k = 3 $$
$$ a = 1 + 0.1 k = 1.3 $$
$$ b = 2 - 0.1 k = 1.7 $$

Рассматриваемый ряд имеет вид:

$$ \sum_{n = 1}^{\infty} \frac{1}{0.2n^2 + 1.3n + 1.7} $$

Рассмотрим последовательно исходный ряд и все улучшения

\section{Анализ исходного ряда}

Погрешность метода заключается в том, что для достижения требуемой точности нам достаточно рассмотреть только $m$ первых членов ряда, так как сумма оставшихся членов будет не превзойдет точности. Оценим сверху остаточную сумму ряда:

$$ err_{m}(m) = \sum_{n = m + 1}^{\infty} \frac{1}{0.2n^2 + 1.3n + 1.7} 
	\le \sum_{n = m + 1}^{\infty} \frac{1}{0.2n^2}
	\le 5 \sum_{n = m + 1}^{\infty} \frac{1}{n^2}
	\le 5 \int_{n = m}^{\infty} \frac{dx}{x^2}
	= 5 (- \frac{1}{x})\biggr|_{m}^{\infty}
	= \frac{5}{m}$$
	
Теперь, допустим мы использем арифметику с плавающей запятой с экспонентой 10 и мантиссой длны $t$. При каждой операции сложения двух чисел $a$ и $b$ накопительную ошибку можно оценить как $|a + b| 10^{-t}$ (складываем два числа без ошибки, затем отбрасываем все, что не влезло в мантиссу). Будем теперь складывать частичный ряд из $m$ членов начиная с самых старших. Оценим сумму, которая будет получатся на каждом этапе для $k \ge 1$:

$$ \sum_{i = k + 1}^{m} \frac{1}{0.2i^2 + 1.3i + 1.7}
	\le \sum_{i = k + 1}^{m} \frac{5}{i^2}
	\le 5 \int_{k}^{m} \frac{dx}{x^2}
	= 5(\frac{1}{k} - \frac{1}{m})
	\le \frac{5}{k} $$
	
Оценим теперь общую погрешность вычисления как сумму погрешностей вычислений на каждом этапе:

$$ err_{c}(m)
	\le \sum_{k = 1}^{m} p_{c}^{(k)}
	= \sum_{k = 1}^{m} 10^{-t} \sum_{i = k}^{m} \frac{1}{0.2i^2 + 1.3i + 1.7}$$
$$ \sum_{k = 1}^{m} 10^{-t} \sum_{i = k}^{m} \frac{1}{0.2i^2 + 1.3i + 1.7}
	\le 10^{-t} ((1 + \frac{5}{2}) + \sum_{k = 2}^{m} \frac{5}{k})$$
$$ 10^{-t} ((1 + \frac{5}{2}) + \sum_{k = 2}^{m} \frac{5}{k})
	\le 10^{-t} 5 (1 + \sum_{k = 2}^{m} \frac{1}{k})
	\le 10^{-t} 5 (1 + \int_{1}^{m} \frac{dx}{x})
	= 10^{-t} 5 (1 + \ln m) $$
	
Теперь общую погрешность можно оценить как

$$ err(m) \le err_{c}(m) + err_{m}(m) = \frac{5}{m} + 10^{-t} 5 (1 + \ln m)
	\le 0.5 \cdot 10^{-4} $$
$$ \frac{1}{m} + 10^{-t} (1 + \ln m) \le 10^{-5} $$

Заметим, что для того, чтобы логарифм числа слагаемых начал быть хотя бы сравнимым с требуемой точностью, при допущении, что $t = 14$ (примерно стандартная двойная точность), необходимо, чтобы выполнилось

$$ ln(m) \sim \frac{10^{14}}{10^5} \sim 10^{9} $$
$$ m \sim e^{10^{9}} $$

Что, мягко говоря, невозможно, поэтому численную погрешность можем не учитывать.

$$ err(m) \approx err_{m}(m) = \frac{5}{m} \le 0.5 \cdot 10^{-4} $$
$$ m \ge 10^5 $$

При данном количестве учтенных слагаемых достигается интересующая точность. Очевидно, что если мы возьмем меньшую точность вычислений, то понадобится больше слагаемых, так как придется уменьшать первое слагаемое. Иногда решений вообще не будет, когда, например, точность вычислений уже меньше, чем требуемая точность.

Минимальное значение мантиссы, при котором решение все-таки существует $t = 7$, в таком случае нам необходимо $m \approx 1.1 \cdot 10^5 $ слагаемых для достижения необходимой точности.

\section{Улучшение исходного ряда}

\subsection{Улучшение рядом $ \sum_{n = 1}^{\infty} \frac{1}{n^2} $}
	
Заметим, что в данном случае слагаемые в сумме будут получатся еще меньше, чем в первом случае, поэтому сразу предположим, что мы используем двойную точностью, которой с лихвой хватат для любых вычислений и вычислительной ошибкой мы можем пренебречь, будем оценивать сразу погрешность метода. Для этого заметим для начала.

$$ \sum_{n = 1}^{\infty} \frac{1}{0.2n^2 + 1.3n + 1.7} =
	\sum_{n = 1}^{\infty} \frac{5}{n^2 + 6.5n + 8.5} =
	\sum_{n = 1}^{\infty} \frac{5}{n^2 + 6.5n + 8.5} + \frac{5}{n^2} - \frac{5}{n^2} = $$
$$ 5 \sum_{n = 1}^{\infty} \frac{1}{n^2} +
	5 \sum_{n = 1}^{\infty} \frac{1}{n^2 + 6.5n + 8.5} - \frac{1}{n^2} = 
	\frac{5\pi^2}{6} +
	5 \sum_{n = 1}^{\infty} \frac{n^2 - (n^2 + 6.5n + 8.5)}{n^2 (n^2 + 6.5n + 8.5)} = $$
$$ \frac{5\pi^2}{6} -
	5 \sum_{n = 1}^{\infty} \frac{6.5n + 8.5}{n^2 (n^2 + 6.5n + 8.5)} = 
	\frac{5\pi^2}{6} -
	\frac{5 \cdot 13}{2} \sum_{n = 1}^{\infty} \frac{n + \frac{17}{13}}{n^2 (n^2 + 6.5n + 8.5)}$$
	
Оценим остаток суммы ряда, если мы взяли только $m$ слагаемых.

$$ \sum_{n = m + 1}^{\infty} \frac{n + \frac{17}{13}}{n^2 (n^2 + 6.5n + 8.5)} \le
	\sum_{n = m + 1}^{\infty} \frac{2n}{n^4} = 
	2 \sum_{n = m + 1}^{\infty} \frac{1}{n^3} \le
	2 \int_{m}^{\infty} \frac{dx}{x^3} =
	-2 \frac{1}{2x^2}\biggr|_{m}^{\infty} =
	\frac{1}{m^2} $$
	
$$ err_{m}(m) =
	\biggr|-\frac{65}{2} \sum_{n = m + 1}^{\infty}
		\frac{n + \frac{17}{13}}{n^2 (n^2 + 6.5n + 8.5)}\biggr| \le
	\frac{65}{2m^2}$$
	
$$ err(m) \approx err_{m}(m) \le \frac{65}{2m^2} \le 0.5 \cdot 10^{-4} $$
$$ m \ge \sqrt{65 * 10^4} \approx 806 $$

Таким образом с помощью ряда $\sum_{n = 1}^{\infty} \frac{1}{n^2}$ мы смогли уменьшить количество слагаемых, которые необходимо посчитать на 2 порядка.
	
\subsection{Улучшение рядом $ \sum_{n = 1}^{\infty} \frac{1}{n^3} $}
	
Улучшим далее наши оценки следующим образом

$$ \frac{5}{2} \sum_{n = 1}^{\infty} \biggr[ \frac{13n + 17}{n^2 (n^2 + 6.5n + 8.5)} - 
	\frac{13}{n^3} + \frac{13}{n^3} \biggr] =$$
$$ \frac{5 \cdot 13}{2} \sum_{n = 1}^{\infty} \frac{1}{n^3} +
	\frac{5}{2} \sum_{n = 1}^{\infty}
	\frac{\cancel{13 n^4} + 17n^3 - \cancel{13 n^4} -
	13 \cdot 6.5 n^3 - 13 \cdot 8.5 n^2}{n^5 (n^2 + 6.5n + 8.5)} = $$
$$ \frac{5 \cdot 13}{2} \zeta(3) +
	\frac{5}{4} \sum_{n = 1}^{\infty}
	\frac{34 n^3 -
	13 \cdot 13 n^3 - 13 \cdot 17 n^2}{n^5 (n^2 + 6.5n + 8.5)} = $$
$$ \frac{5 \cdot 13}{2} \zeta(3) -
	\frac{5}{4} \sum_{n = 1}^{\infty}
	\frac{135 n^3 + 221 n^2}{n^5 (n^2 + 6.5n + 8.5)}$$
	
Где $\zeta(3)$ --- дзета-функция Рейманна.
	
Аналогичным образом посчитаем погрешность метода через остаток ряда

$$ err_{m}(m) = \biggr| - \frac{5}{4} \sum_{n = m + 1}^{\infty}
	\frac{135 n^3 + 221 n^2}{n^5 (n^2 + 6.5n + 8.5)} \biggr| \le
	\frac{5 \cdot 135 \cdot 2}{4} \sum_{n = m + 1}^{\infty} \frac{1}{n^4} \le
	\frac{675}{6 m^3} \le 0.5 \cdot 10^{-4} $$
$$ m \ge \sqrt[3]{\frac{135 \cdot 10^5}{6}} \approx 131 $$
		
\subsection{Улучшение рядом $ \sum_{n = 1}^{\infty} \frac{1}{n^4} $}
		
Улучшим далее наши оценки следующим образом

$$ \frac{5}{4} \sum_{n = m + 1}^{\infty}
	\biggr[ \frac{135 n^3 + 221 n^2}{n^5 (n^2 + 6.5n + 8.5)} - 
	\frac{135}{n^4} + \frac{135}{n^4} \biggr] =$$
$$ \frac{5 \cdot 135}{4} \sum_{n = 1}^{\infty} \frac{1}{n^4} +
	\frac{5}{8} \sum_{n = 1}^{\infty}
	\frac{\cancel{270 n^7} + 442 n^6 - \cancel{270 n^7} -
		135 \cdot 13 n^6 - 135 \cdot 17 n^5}{n^9 (n^2 + 6.5n + 8.5)} = $$
$$ \frac{5 \cdot 135}{4} \frac{\pi^4}{90} -
	\frac{5}{8} \sum_{n = 1}^{\infty}
	\frac{1313 n^6 + 2295 n^5}{n^9 (n^2 + 6.5n + 8.5)} = $$
$$ \frac{15 \pi^4}{8} -
	\frac{5}{8} \sum_{n = 1}^{\infty}
	\frac{1313 n^6 + 2295 n^5}{n^9 (n^2 + 6.5n + 8.5)} $$

Аналогичным образом посчитаем погрешность метода через остаток ряда

$$ err_{m}(m) =
	\biggr| - \frac{5}{8} \sum_{n = m + 1}^{\infty}
		\frac{1313 n^6 + 2295 n^5}{n^9 (n^2 + 6.5n + 8.5)} \biggr| \le
	\frac{5 \cdot 1313 \cdot 2}{8} \sum_{n = m + 1}^{\infty} \frac{1}{n^5} \le
	\frac{5 \cdot 1313 \cdot 2}{8 \cdot 4 \cdot m^4} = 
	\frac{5 \cdot 1313}{16 m^4} \le 0.5 \cdot 10^{-4} $$
$$ m \ge \sqrt[4]{ \frac{1313 \cdot 10^{5}}{16} } \approx 54 $$

\section{Вычислительный эксперимент}

\begin{tabular}{ | l | l | l | l | l | }
	\hline
	Кол-во слагаемых
		& Исходный ряд
		& Улучшение (1)
		& Улучшение (2)
		& Улучшение (3) \\ \hline
	20
		& 1.1837195539
		& 1.4275736686
		& \underline{1.3}889291796
		& \underline{1.39}54506421 \\ \hline
	100 
		& \underline{1.3}462762433
		& \underline{1.39}60270766
		& \underline{1.3944}182454
		& \underline{1.39447}36572 \\ \hline
	1000
		& \underline{1.3}894903965
		& \underline{1.3944}878974
		& \underline{1.39447}16636
		& \underline{1.39447}17198 \\ \hline
	рассчитанное
		& \underline{1.3944}217214
		& \underline{1.3944}965344
		& \underline{1.3944}476502
		& \underline{1.3944}934544 \\ \hline
\end{tabular}

\section{Выводы}

Практические результаты сошлись с аналитическими расчетам, однако оказалось, что при увеличении количества слагаемых в исходном ряду точность не сильно увеличивается. Скорее всего, ошибка вошла в расчеты, связанные с точностью вычисления отдельного слагаемого в сумме. Дело в том, что для того, чтобы вычислить очередной член ряда, необходимо сложить достаточно большие числа, значительно потеряв в точности, прежде чем брать обратное к этому числу. В данном случае вольно сделанные допущения не сделали результат в корне неверными, однако такое могло случится.

\end{document}
