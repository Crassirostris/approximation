\documentclass[11pt,a4paper,oneside]{article}

%\usepackage{pscyr}
\usepackage[T2A]{fontenc}
\usepackage[utf8x]{inputenc}
\usepackage[english,russian]{babel}
\usepackage{expdlist}
\usepackage[dvips]{graphicx}
\usepackage{amsmath}
\usepackage[makeroom]{cancel}
\usepackage{svg}
\usepackage[top=1in, bottom=1in, left=1in, right=1in]{geometry}

\begin{document}

\begin{center}
	{Вяцков Михаил, КН-401}
	
	{\huge \bf Лабораторная работа №1}
\end{center}

По условиям лабораторной

$$ k = 3 $$
$$ a = 1 + 0.1 k = 1.3 $$
$$ b = 2 - 0.1 k = 1.7 $$

Рассматриваемый ряд имеет вид:

$$ \sum_{n = 1}^{\infty} \frac{1}{0.2n^2 + 1.3n + 1.7} $$

Рассмотрим последовательно исходный ряд и все улучшения

\section{Вычисления с погрешностью}

Теперь, допустим мы использем арифметику с плавающей запятой с экспонентой 10 и мантиссой длны $t$. Допустим мы работаем только с положительными числами. Пусть все числа изначально хранятся с относительной точностью $\varepsilon = 5 \cdot 10^{-t}$. Пусть есть два числа $x$ и $y$, вычисленных с относительной точностью $\varepsilon_x$ и $\varepsilon_y$. Тогда в машинном представлении на самом деле хранятся числа $x (1 + \delta_x)$ и $y (1 + \delta_y)$, где $|\delta_x| \le \varepsilon_x$ и $|\delta_y| \le \varepsilon_y$.

Рассмотрим сумму:

$$ x (1 + \delta_x) + y (1 + \delta_y) = (x + y) (1 + \frac{x}{x + y} \delta_x + \frac{y}{x + y} \delta_y) $$
$$ \biggr|\frac{x}{x + y} \delta_x + \frac{y}{x + y} \delta_y\biggr|
\le \biggr|\frac{x}{x + y} \delta_x\biggr| + \biggr|\frac{y}{x + y} \delta_y\biggr|
\le \frac{x}{x + y} \varepsilon_x + \frac{y}{x + y} \varepsilon_y $$
$$ \frac{x}{x + y} \varepsilon_x + \frac{y}{x + y} \varepsilon_y
\le \frac{x}{x + y} \max \{\varepsilon_x, \varepsilon_y\}
+ \frac{y}{x + y} \max \{\varepsilon_x, \varepsilon_y\}
= \max \{\varepsilon_x, \varepsilon_y\} \biggr(\frac{x}{x + y} + \frac{y}{x + y}\biggr)$$
$$ \frac{x}{x + y} + \frac{y}{x + y} = \frac{x + y}{x + y} = 1
\implies \biggr|\frac{x}{x + y} \delta_x + \frac{y}{x + y} \delta_y\biggr|
\le \max \{\varepsilon_x, \varepsilon_y\} $$

Таким образом, точно вычисленная сумма приближенных значений будет иметь относительную погрешность $\max \{\varepsilon_x, \varepsilon_y\}$, но кроме того, нам необходимо теперь сохранить это значение, а значит в худшем случае к относительной погрешности добавится еще $\varepsilon$.

Рассмотрим произведение:

$$ x (1 + \delta_x) y (1 + \delta_y) = xy (1 + \delta_x + \delta_y + \delta_x \delta_y) $$
$$ |\delta_x + \delta_y + \delta_x \delta_y|
	\le \varepsilon_x + \varepsilon_y + \varepsilon_x \varepsilon_y $$
	
Опять же, после вычисления точного значения может произойти округление, которое добавит к относительное погрешности $\varepsilon$.

Рассмотрим отношение:

$$ \frac{x (1 + \delta_x)}{y (1 + \delta_y)} = \frac{x}{y} \frac{1 + \delta_x}{1 + \delta_y} $$

Разложим знаменатель второй дроби в ряд Тейлора:

$$ (1 + \delta_y)^{-1} = 1 - \delta_y + \delta_y^2 - \delta_y^3 + .. = 1 - \delta_y (1 - \delta_y + \delta_y^2 - ..) = 1 - \frac{\delta_y}{1 + \delta_y} $$
$$ \frac{x}{y} \frac{1 + \delta_x}{1 + \delta_y}
	= \frac{x}{y} (1 + \delta_x) (1 - \frac{\delta_y}{1 + \delta_y})
	= \frac{x}{y} (1 
		+ \delta_x - \frac{\delta_y}{1 + \delta_y} - \frac{\delta_x \delta_y}{1 + \delta_y}) $$
$$ \biggr|\delta_x - \frac{\delta_y}{1 + \delta_y} - \frac{\delta_x \delta_y}{1 + \delta_y}\biggr|
	\le \varepsilon_x
		+ \frac{\varepsilon_y}{|1 + \delta_y|}
		+ \frac{\varepsilon_x \varepsilon_y}{|1 + \delta_y|} $$
		
Пусть $\max\{\varepsilon_x, \varepsilon_y\} < 1$, тогда

$$ |1 + \delta_y| \ge 1 + \delta_y \ge 1 - \varepsilon_y $$
$$ \frac{1}{|1 + \delta_y|} \le \frac{1}{1 - \varepsilon_y} $$
$$ \varepsilon_x
		+ \frac{\varepsilon_y}{|1 + \delta_y|}
		+ \frac{\varepsilon_x \varepsilon_y}{|1 + \delta_y|}
	\le \varepsilon_x
		+ \frac{\varepsilon_y}{1 - \varepsilon_y}
		+ \frac{\varepsilon_x \varepsilon_y}{1 - \varepsilon_y} $$

\section{Анализ исходного ряда}

Погрешность метода заключается в том, что для достижения требуемой точности нам достаточно рассмотреть только $m$ первых членов ряда, так как сумма оставшихся членов будет не превзойдет точности. Оценим сверху остаточную сумму ряда:

$$ err_{m}(m) = \sum_{n = m + 1}^{\infty} \frac{1}{0.2n^2 + 1.3n + 1.7} 
	\le \sum_{n = m + 1}^{\infty} \frac{1}{0.2n^2}
	\le 5 \sum_{n = m + 1}^{\infty} \frac{1}{n^2}
	\le 5 \int_{n = m}^{\infty} \frac{dx}{x^2}
	= 5 (- \frac{1}{x})\biggr|_{m}^{\infty}
	= \frac{5}{m}$$
	
Пусть числа хранятся с относительной погрешностью $\varepsilon$, посчитаем относительную погрешность вычисления одного слагаемого, обозначим относительную погрешность величины $x$ как $\delta x$. Заметим, что целое число, умещающееся в размер мантиссы, имеет нулевую погрешность, как, например, 1 или $n$.

$$ \delta (0.2 \cdot n) = \varepsilon + \varepsilon = 2 \varepsilon $$
$$ \delta ((0.2 \cdot n) + 1.3) = 2 \varepsilon + \varepsilon = 3 \varepsilon $$
$$ \delta (n \cdot (0.2 n + 1.3)) = 4 \varepsilon + \varepsilon = 5 \varepsilon $$
$$ \delta ((n (0.2 n + 1.3)) + 1.7) = 5 \varepsilon + \varepsilon = 6 \varepsilon $$
$$ \delta \biggr( \frac{1}{n (0.2 n + 1.3) + 1.7} \biggr)
	= \frac{6 \varepsilon}{1 - 6 \varepsilon} + \varepsilon $$
	
Если мы складываем $m$ слагаемых, то будет еще $m - 1$ сложений, каждое из которых добавит к относительной погрешности еще столько же $\varepsilon$. Итоговая вычислительная погрешность, учитывая относительный характер погрешностей, которые мы оценивали и тот факт, что итоговая сумма не превосходит 2, будет выглядеть таким образом:

$$ err_{c}(m) = 2 \biggr(m\varepsilon + \frac{6 \varepsilon}{1 - 6 \varepsilon}\biggr) $$
	
Теперь общую погрешность можно оценить как

$$ err(m) \le err_{c}(m) + err_{m}(m)
	= \frac{5}{m} + 2m \varepsilon + \frac{12 \varepsilon}{1 - 6 \varepsilon}
	\le 0.5 \cdot 10^{-4} $$
$$ \frac{5}{m} + 10 m 10^{-t} + \frac{60}{10^{t} - 30} \le 5 \cdot 10^{-5} $$
$$ \frac{1}{m} + 2 m 10^{-t} + \frac{12}{10^{t} - 30} \le \cdot 10^{-5} $$

Минимальный размер мантиссы, при котором решение все-таки существует $t = 11$, в таком случае нам необходимо $m \approx 1.1 \cdot 10^5 $ слагаемых для достижения необходимой точности. В условиях двойной точности, однако, согласной этой формуле, достаточно $10^5 + 1$ слагаемых. Максимальная точность, которую можно достигнуть --- $0.5 \cdot 10^{-6}$, для этого необходимо $m \approx 1.1 \cdot 10^8$ слагаемых.

\section{Улучшение исходного ряда}

\subsection{Улучшение рядом $ \sum_{n = 1}^{\infty} \frac{1}{n^2} $}
	
Оценим погрешность метода. Для этого заметим для начала.

$$ \sum_{n = 1}^{\infty} \frac{1}{0.2n^2 + 1.3n + 1.7} =
	\sum_{n = 1}^{\infty} \frac{5}{n^2 + 6.5n + 8.5} =
	\sum_{n = 1}^{\infty} \frac{5}{n^2 + 6.5n + 8.5} + \frac{5}{n^2} - \frac{5}{n^2} = $$
$$ 5 \sum_{n = 1}^{\infty} \frac{1}{n^2} +
	5 \sum_{n = 1}^{\infty} \frac{1}{n^2 + 6.5n + 8.5} - \frac{1}{n^2} = 
	\frac{5\pi^2}{6} +
	5 \sum_{n = 1}^{\infty} \frac{n^2 - (n^2 + 6.5n + 8.5)}{n^2 (n^2 + 6.5n + 8.5)} = $$
$$ \frac{5\pi^2}{6} -
	5 \sum_{n = 1}^{\infty} \frac{6.5n + 8.5}{n^2 (n^2 + 6.5n + 8.5)} = 
	\frac{5\pi^2}{6} -
	\frac{5 \cdot 13}{2} \sum_{n = 1}^{\infty} \frac{n + \frac{17}{13}}{n^2 (n^2 + 6.5n + 8.5)}$$
	
Оценим остаток суммы ряда, если мы взяли только $m$ слагаемых.

$$ \sum_{n = m + 1}^{\infty} \frac{n + \frac{17}{13}}{n^2 (n^2 + 6.5n + 8.5)} \le
	\sum_{n = m + 1}^{\infty} \frac{n + \frac{17}{13}}{n^4} = 
	\sum_{n = m + 1}^{\infty} \frac{1}{n^3} + \frac{17}{13 n^4} \le
	\int_{m}^{\infty} \biggr(\frac{1}{x^3} + \frac{17}{13 x^4}\biggr)dx $$
$$ \int_{m}^{\infty} \biggr(\frac{1}{x^3} + \frac{17}{13 x^4}\biggr) dx
	= \biggr( - \frac{1}{2x^2} - \frac{17}{39 x^3} \biggr) \biggr|_{m}^{\infty} =
	\frac{1}{2 m^2} + \frac{17}{39 m^3} $$
	
$$ err_{m}(m) =
	\biggr|-\frac{65}{2} \sum_{n = m + 1}^{\infty}
		\frac{n + \frac{17}{13}}{n^2 (n^2 + 6.5n + 8.5)}\biggr| \le
	\frac{65}{4m^2} + \frac{17 \cdot 65}{78m^3}$$
	
Можно организовать вычисления таким образом, чтобы позже показать, что

$$ \delta \biggr( \frac{n + \frac{17}{13}}{n^2 (n^2 + 6.5n + 8.5)} \biggr)
	\le \frac{7 \varepsilon}{1 - 7 \varepsilon} + \varepsilon $$
	
По старой схеме вычислим ошибку
	
$$ err(m) \le err_{m}(m) + err_{c}(m)
	\le \frac{65}{4m^2} + \frac{17 \cdot 65}{78m^3}
		+ 2 m \varepsilon + \frac{14 \varepsilon}{1 - 7 \varepsilon}
	\le 0.5 \cdot 10^{-4} $$
$$ \frac{13}{4m^2} + \frac{17 \cdot 13}{78m^3}
		+ 2 m 10^{-t} + \frac{14}{10^{t} - 35} \le 10^{-5} $$

Минимальная точность, которой достаточно, чтобы это уравнение имело решение --- 9 знаков в мантиссе. В этом случае необходимо 610 слагаемых. В случае двойной точности достаточно 570 слагаемых.
	
\subsection{Улучшение рядом $ \sum_{n = 1}^{\infty} \frac{1}{n^3} $}
	
Улучшим далее наши оценки следующим образом

$$ \frac{5}{2} \sum_{n = 1}^{\infty} \biggr[ \frac{13n + 17}{n^2 (n^2 + 6.5n + 8.5)} - 
	\frac{13}{n^3} + \frac{13}{n^3} \biggr] =$$
$$ \frac{5 \cdot 13}{2} \sum_{n = 1}^{\infty} \frac{1}{n^3} +
	\frac{5}{2} \sum_{n = 1}^{\infty}
	\frac{\cancel{13 n^4} + 17n^3 - \cancel{13 n^4} -
	13 \cdot 6.5 n^3 - 13 \cdot 8.5 n^2}{n^5 (n^2 + 6.5n + 8.5)} = $$
$$ \frac{5 \cdot 13}{2} \zeta(3) +
	\frac{5}{4} \sum_{n = 1}^{\infty}
	\frac{34 n^3 -
	13 \cdot 13 n^3 - 13 \cdot 17 n^2}{n^5 (n^2 + 6.5n + 8.5)} = $$
$$ \frac{5 \cdot 13}{2} \zeta(3) -
	\frac{5}{4} \sum_{n = 1}^{\infty}
	\frac{135 n^3 + 221 n^2}{n^5 (n^2 + 6.5n + 8.5)}$$
	
Где $\zeta(3)$ --- дзета-функция Рейманна.
	
Аналогичным образом посчитаем погрешность метода через остаток ряда

$$ err_{m}(m) = \biggr| - \frac{5}{4} \sum_{n = m + 1}^{\infty}
	\frac{135 n^3 + 221 n^2}{n^5 (n^2 + 6.5n + 8.5)} \biggr| \le
	\frac{5 \cdot 135 \cdot 2}{4} \sum_{n = m + 1}^{\infty} \frac{1}{n^4} \le
	\frac{675}{6 m^3} \le 0.5 \cdot 10^{-4} $$
$$ m \ge \sqrt[3]{\frac{135 \cdot 10^5}{6}} \approx 131 $$
		
\subsection{Улучшение рядом $ \sum_{n = 1}^{\infty} \frac{1}{n^4} $}
		
Улучшим далее наши оценки следующим образом

$$ \frac{5}{4} \sum_{n = m + 1}^{\infty}
	\biggr[ \frac{135 n^3 + 221 n^2}{n^5 (n^2 + 6.5n + 8.5)} - 
	\frac{135}{n^4} + \frac{135}{n^4} \biggr] =$$
$$ \frac{5 \cdot 135}{4} \sum_{n = 1}^{\infty} \frac{1}{n^4} +
	\frac{5}{8} \sum_{n = 1}^{\infty}
	\frac{\cancel{270 n^7} + 442 n^6 - \cancel{270 n^7} -
		135 \cdot 13 n^6 - 135 \cdot 17 n^5}{n^9 (n^2 + 6.5n + 8.5)} = $$
$$ \frac{5 \cdot 135}{4} \frac{\pi^4}{90} -
	\frac{5}{8} \sum_{n = 1}^{\infty}
	\frac{1313 n^6 + 2295 n^5}{n^9 (n^2 + 6.5n + 8.5)} = $$
$$ \frac{15 \pi^4}{8} -
	\frac{5}{8} \sum_{n = 1}^{\infty}
	\frac{1313 n^6 + 2295 n^5}{n^9 (n^2 + 6.5n + 8.5)} $$

Аналогичным образом посчитаем погрешность метода через остаток ряда

$$ err_{m}(m) =
	\biggr| - \frac{5}{8} \sum_{n = m + 1}^{\infty}
		\frac{1313 n^6 + 2295 n^5}{n^9 (n^2 + 6.5n + 8.5)} \biggr| \le
	\frac{5 \cdot 1313 \cdot 2}{8} \sum_{n = m + 1}^{\infty} \frac{1}{n^5} \le
	\frac{5 \cdot 1313 \cdot 2}{8 \cdot 4 \cdot m^4} = 
	\frac{5 \cdot 1313}{16 m^4} \le 0.5 \cdot 10^{-4} $$
$$ m \ge \sqrt[4]{ \frac{1313 \cdot 10^{5}}{16} } \approx 54 $$

\section{Вычислительный эксперимент}

\begin{tabular}{ | l | l | l | l | l | }
	\hline
	Кол-во слагаемых
		& Исходный ряд
		& Улучшение (1)
		& Улучшение (2)
		& Улучшение (3) \\ \hline
	20
		& 1.1837195539
		& 1.4275736686
		& \underline{1.3}889291796
		& \underline{1.39}54506421 \\ \hline
	100 
		& \underline{1.3}462762433
		& \underline{1.39}60270766
		& \underline{1.3944}182454
		& \underline{1.39447}36572 \\ \hline
	1000
		& \underline{1.3}894903965
		& \underline{1.3944}878974
		& \underline{1.39447}16636
		& \underline{1.39447}17198 \\ \hline
	рассчитанное
		& \underline{1.3944}217214
		& \underline{1.3944}965344
		& \underline{1.3944}476502
		& \underline{1.3944}934544 \\ \hline
\end{tabular}

\section{Выводы}

Практические результаты сошлись с аналитическими расчетам, однако оказалось, что при увеличении количества слагаемых в исходном ряду точность не сильно увеличивается. Скорее всего, ошибка вошла в расчеты, связанные с точностью вычисления отдельного слагаемого в сумме. Дело в том, что для того, чтобы вычислить очередной член ряда, необходимо сложить достаточно большие числа, значительно потеряв в точности, прежде чем брать обратное к этому числу. В данном случае вольно сделанные допущения не сделали результат в корне неверными, однако такое могло случится.

\end{document}
